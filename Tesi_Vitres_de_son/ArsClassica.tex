%!TEX TS-program = pdflatex
%!TEX encoding = UTF-8 Unicode
%!TEX root = ArsClassica.tex

\documentclass[a4paper,
		       titlepage,
               headinclude,
               footinclude,
               BCOR5mm,
               numbers=noenddot,
               cleardoublepage=empty,
               tablecaptionabove
               ]{scrreprt}

\usepackage[T1]{fontenc}

\usepackage[english,
            italian
            ]{babel}
            
\usepackage{amsmath,amssymb}
\usepackage{indentfirst}
\usepackage[style=philosophy-modern,hyperref]{biblatex}
\usepackage{chngpage}
\usepackage{calc}
\usepackage{listings}
\usepackage{graphicx}
\usepackage{subfig}
\usepackage{lipsum}
\usepackage{shapepar}
\usepackage{pifont}
\usepackage[eulerchapternumbers,subfig,beramono,eulermath,pdfspacing,listings]{classicthesis}
\usepackage{arsclassica}

\input{arsclassica-settings}

\begin{document}
\pagenumbering{roman}
\pagestyle{plain}
%!TEX TS-program = pdflatex
%!TEX encoding = UTF-8 Unicode
%!TEX root = ../2018-03-26-papa-vitres-de-son.tex

%*******************************************************
% Titlepage
%*******************************************************

\begin{titlepage}
\pdfbookmark{Titlepage}{Titlepage}
\changetext{}{}{}{((\paperwidth  - \textwidth) / 2) - \oddsidemargin - \hoffset - 1in}{}
    \begin{center}

\begin{table}[htp]
\begin{center}
\begin{tabular}{rl}
\multirow{ 2}{*}{\includegraphics[scale=0.105]{Conservatorio.pdf}} & \LARGE \spacedlowsmallcaps{Conservatorio di Musica S. Cecilia di Roma} \\ \cline{2-2}
& \spacedlowsmallcaps{DIPARTIMENTO DI NUOVE TECNOLOGIE E LINGUAGGI MUSICALI} \\
& \spacedlowsmallcaps{SCUOLA DI MUSICA ELETTRONICA} \\
\end{tabular}
\end{center}
\label{default}
\end{table}%
        
%        \LARGE \spacedlowsmallcaps{Conservatorio di Musica S. Cecilia di Roma}
%        
%        \bigskip
%        
%        \hrule
%        
%        \bigskip
%        
%        \large \spacedlowsmallcaps{DIPARTIMENTO DI NUOVE TECNOLOGIE E LINGUAGGI MUSICALI}
%        
%        \spacedlowsmallcaps{SCUOLA DI MUSICA ELETTRONICA}
        
		\vfill
        
        \spacedlowsmallcaps{CORSO DI DIPLOMA ACCADEMICO DI PRIMO LIVELLO IN}
                       
        \LARGE \spacedlowsmallcaps{MUSICA ELETTRONICA}

        \vfill ~ \vfill

        \LARGE {\color{Maroon}\spacedallcaps{\myTitle}}
        
        \large \mySubTitle 
        
        \vfill
        
        \normalsize Candidato: \\
        \Large \spacedlowsmallcaps{\myName}
        
        \normalsize Matricola: \\
        \Large \spacedlowsmallcaps{2945TR}
        
        \bigskip
        
        \normalsize Relatore: \\
        \Large \spacedlowsmallcaps{Michelangelo Lupone}

        \vfill ~ \vfill ~ \vfill
        
        \normalsize Anno Accademico: \\
        \Large \spacedlowsmallcaps{2016-2017}


%        \includegraphics[width=0.7\textwidth]{TFZSuperEllisse} \\ \bigskip
                   

    \end{center}        

\end{titlepage} 
%!TEX TS-program = pdflatex
%!TEX encoding = UTF-8 Unicode
%!TEX root = ../2018-03-26-papa-vitres-de-son.tex

%*******************************************************
% Titleback
%*******************************************************
\thispagestyle{empty}
\pdfbookmark{Titleback}{Titleback}

\hfill

\vspace{\stretch{2}}

\begin{center}
Lorenzo Pantieri \\
\smallskip
\textit{The \arsclassica{} package}\\
\smallskip
Copyright\,\textcopyright\ 2008-2017
\end{center}
\vspace{\stretch{1}}

\medskip

\noindent\textsf{\spacedlowsmallcaps{Titleback}} \\
\noindent
This document was written with \LaTeX{} on Mac using \arsclassica, a reworking of the \classicthesis{} style designed by Andr\'e Miede, inspired to the masterpiece \emph{The Elements of Typographic Style} by Robert Bringhurst. 

\bigskip

\noindent
\textsf{\spacedlowsmallcaps{Contacts}}

\noindent
{\raisebox{-0.33ex}{\ding{43}}}\,\mail{lorenzo.pantieri@gmail.com}
\cleardoublepage
%!TEX TS-program = pdflatex
%!TEX encoding = UTF-8 Unicode
%!TEX root = ../2018-03-26-papa-vitres-de-son.tex

%*******************************************************
% Introduzione
%*******************************************************
\pdfbookmark{Introduzione}{Introduzione}

\chapter*{Introduzione}

Ogni creazione artistica è legata imprescindibilmente sia alla carriera accademica che alla storia personale di chi la scrive.

Nel corso della stesura di questa tesi, ho cercato di esprimere il mio processo compositivo e creativo. La forma del testo, per una resa più diretta e scorrevole, è in forma romanzata, è una storia. Cercando in qualche modo di sfuggire da una semplice compilazione di liste sulle tecniche e sui materiali sonori utilizzati.

Questa composizione è più un sentiero da intraprendere ed è forse questo il lavoro che un compositore elettroacustico fa giornalmente: non la sola scrittura notazionale e simbolica, ma una continua ricerca di fenomeni che incarnino il proprio vissuto e nel quale un artista può identificarsi, facendoli suoi e riportandoli in musica, creando una propria identità artistica e formale. Come scrisse Kandisky nell'introduzione di \textit{Punto e linea nel piano}: 

\begin{small}
\begin{quotation}
\textit{Di ogni fenomeno si può fare esperienza in due modi. \\
Questi due modi non sono arbitrari ma connessi ai fenomeni -  essi vengono derivati dalla natura dei fenomeni, da due proprietà degli stessi: \\ 
\centerline{esterno - interno\footnote{Wassily Kandisky \textit{Punto e linea nel piano}, Adelphi Edizioni, Roma, 1968}.}}
\end{quotation}
\end{small}

Muovere \textit{interna}mente le corde della propria coscienza e delle proprie esperienze, provando a far vibrare ogni suggestione verso qualcosa di nuovo. Per "qualcosa di nuovo" non intendo un oggetto o una scrittura mai letta o vista prima. Il nuovo, penso, sia semplicemente l'unione di tecniche e di procedimenti fisici che per la prima volta subiscono quella successione di eventi nel tempo. Il tempo crea il nuovo nelle svariate possibilit\`a che la vita ci propone: verso un \textit{esterno} dove abbiamo la possibilit\`a di essere semplici ascoltatori o fautori dei cambiamenti che avvengono in tale realtà. 
\pagestyle{scrheadings}
%!TEX TS-program = pdflatex
%!TEX encoding = UTF-8 Unicode
%!TEX root = ../2018-03-26-papa-vitres-de-son.tex

%*******************************************************
% Contents
%*******************************************************

\phantomsection
\pdfbookmark{\contentsname}{tableofcontents}
\setcounter{tocdepth}{2}
\tableofcontents
\markboth{\spacedlowsmallcaps{\contentsname}}{\spacedlowsmallcaps{\contentsname}} 
\cleardoublepage
\pagenumbering{arabic}
%\input{Chapters/01-sinossi}
%\input{Chapters/02-spire}
%\input{Chapters/03-performance}
%\input{Chapters/04-composizione}
%\input{Chapters/05-materiali}
%\input{Chapters/06-diffusione}
%\input{Chapters/07-conclusioni}
\clearpage
%!TEX TS-program = pdflatex
%!TEX encoding = UTF-8 Unicode
%!TEX root = ../2018-03-26-papa-vitres-de-son.tex

%*******************************************************
% Bibliography
%*******************************************************
\nocite{*}
\printbibliography
\end{document}
