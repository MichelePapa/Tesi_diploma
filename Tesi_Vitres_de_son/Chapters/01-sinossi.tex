% !TEX TS-program = pdflatex
% !TEX root = ../ArsClassica.tex

%************************************************
\chapter{Sinossi}
\label{chp:Sinossi}
%************************************************

\textit{Vitres de Son - Come un rosone nel cuore di un tempio immenso} non � solo una composizione legata alla creazione e alla ricerca su un nuovo \textit{oggetto sonoro}, ma fa parte di un percorso compositivo e creativo che dura ormai da quattro anni. \\
Il titolo e il sottotitolo dell'opera sono rispettivamente il titolo e un verso di due poesie del drammaturgo, poeta e attore Antonin Artaud: \textit{Vitres de Son} e \textit{In Sogno}. Entrambe le poesie sono presenti in \textit{Poesie della crudelt�}\footnote{Antonin Artaud, \textit{Poesie della crudelt�} (a cura di P. Di Palmo), Stampa alternativa, Roma 2002, a cura di P. Di Palmo. Pubblicata per la prima volta nel 1925 \\}. \\
Artaud era artista ai margini, esponente del movimento surrealista e molto discusso dalla critica per le sue idee estremiste riguardanti il teatro, la messa in scena  e la modalit� di diffusione della sua arte:
\begin{small}
\begin{quotation}
\textit{Nell'epoca di confusione in cui viviamo, epoca colma di bestemmie e delle fosforescenze di un rinnegamento infinito, in cui tutti i valori sia artistici che morali sembrano sprofondare in un abisso senza altro esempio in nessun epoca dello spirito, ho avuto la debolezza di credere che avrei potuto fare un teatro, che avrei potuto almeno avviare il tentativo di ridare vita al valore universalmente disprezzato del teatro, ma la stupidit� di alcuni, la malafede e la spregevole canaglieria di altri me ne hanno distolto per sempre. } [...] \footnote{Antonin Artaud, \textit{Il teatro e il suo doppio}, Einaudi Autore, Roma 1968 \\}
\end{quotation}
\end{small}
L'artista spinge verso una critica sociale che lo porter� ai margini della societ� e nella quale mi rispecchio. Inoltre, la figura del drammaturgo si pu� accostare a quella dei \textit{clerici vagantes}\footnote{Furono cos� chiamati per tutto il sec. XII e il XIII quei poeti, che, vivendo al margine della chiesa, vagavano per le universit�, le citt� e le corti, spesso confusi con i giullari, di cui condividevano la vita errabonda e l'indole artistica. [...] \\ \textit{Enciclopedia Treccani} Edizione 2018. \\}. Sia Artaud che i clierici erano, infatti, personaggi solitari, irrequieti, artisti vaganti che risentirono quel potente risveglio intellettuale e politico della loro epoca (in entrambe le epoche), rispecchiandone le condizioni sociali ma soprattutto la fisionomia morale. L'artista � sempre stato un personaggio ai margini e soprattuto Artaud ha ricevuto non poche cure psichiatriche dopo essere arrivato a deliri tali da fermarlo anche nella sua produzione artistica. Per fortuna, le poesie utilizzate per la composizione del mio lavoro, sono legate al primo periodo artistico, quello giovanile, dove ancora riesce ad esprimere un proprio universo immaginifico. Il suo � un paesaggio fatto di personaggi e suoni, potrei azzardare quasi un \textit{paesaggio sonoro}, alla maniera di Schafer, dove ogni cosa, ogni suono, pu� diventare personaggio.
\begin{quotation}
[...] \\
come un rosone nel cuore di un tempio immenso. \\
E l� ascolteremo la cadenza immortale \\
delle linee e dei corpi ritmati \\
e di gotiche balaustre profumate \\
dalla dolcezza dei corpi amati \\
dagli uomini con grandi anime cadenzate, \\
\centerline{dai poeti profumati\footnote{Antonin Artaud, \textit{In Sogno} in Poesie della crudelt� \\}.}
\end{quotation}

Ecco, qui entra in gioco la fase di analisi e costruzione compositiva: l'unione di una visione - all'esterno - ascoltando, ad esempio, delle risonanze create da dita sui del rosone nel cuore del tempio immenso: riecheggiano - all'interno \footnote{qui faccio riferimento a \textit{Punto e linea nel piano} di Kandisky \\}- di un ambiente vuoto.
Al margine delle balaustre, assieme ad Artaud e i clerici vagantes, osserviamo i cambiamenti che avvengono sul linguaggio e sulla sua forma, sull'ambiente. \\
\begin{quotation}
\textbf{Vitres de Son} \\
Vitres de son o� virent les astres \\
verres o� cuisent les cerveaux \\
le ciel fourmillant d'impudeurs \\
d�vore la nudit� des astres. \\ \\ 
Un lait bizarre et v�h�ment \\
fourmille au fond du firmament \\
un escargot monte et d�range \\
la placidit� des nuages. \\ \\ 
D�lices et rages, le ciel entier \\
lance sur nous comme un nuage \\
un tourbillon d'ailes sauvages \\
torrentielles d'obsc�nit�s\footnote{Artaud, \textit{Vitres de Son} in Poesie della crudelt� \\}. \\
\end{quotation}
I \textit{Vetri di suono} diffondono un materiale acustico fascinoso sia nella forma di diffusione elettroacustica che nella tipologia del gesto: mentre l'eccitazione delle armoniche e la componente elettronica, ci spingono verso l'alto, la vibrazione delle sub-armoniche di grandi molle \footnote{vedi cap. 2 \\} ci terr� con i piedi ben saldi al pavimento antistante alla cattedrale, per me astrazione immaginifica della sala da concerto. \\
Davanti a questo universo fatto di paesaggi e personaggi sonori, entra in causa la composizione. Il processo compositivo legato alle immagini e al ritmo dei versi di Artaud sar� il fulcro dominante dell'andamento formale di \textit{Vitres de Son - Come un rosone nel cuore di un tempio immenso}.