% !TEX TS-program = pdflatex
% !TEX root = ../ArsClassica.tex

%************************************************
\chapter{Sistema di diffusione)}
\label{chp:Sistema di diffusione}
%************************************************

\epigraph{Come un rosone nel cuore di un tempio immenso}{\textit{Antonin Artaud}}

\section{Sistema di ripresa}
\addcontentsline{toc}{section}{Sistema di ripresa}
La diffusione audio avverr� tramite l'utilizzo di un sistema di diffusione omnidirezionale che render� possibile la diffusione omogenea del materiale acustico ed elettronico prodotto da Sp.i.r.e.. \\
Strumenti utilizzati:
\begin{center}
\begin{minipage}[c]{1.\textwidth}
\includegraphics[width=1.\textwidth]{Diffusione.jpg}
\end{minipage}
\end{center}
\begin{itemize}
	\item{Sp.i.r.e.}
\end{itemize}

Tutti gli studi sono stati fatti su acciaio armonico o acciaio inox. Due sono i fattori che regolano il funzionamento della molla a trazione:
\begin{enumerate}
\item{Diametro del filo}
\item{Larghezza del diametro esterno (\textit{spira})}
\end{enumerate}
Il diametro del filo (1.) unito alla larghezza del diametro esterno (2.) rendono possibile il cambiamento della qualit� della flessione della molla. Anche il numero di spire agisce sulla flessione della molla.

\section{Diffusione audio e B-Format}
\addcontentsline{toc}{section}{Diffusione audio e B-Format}

Per la costruzione del sistema di diffusione ho utilizzo gli scritti introduttivi che sono allegati a molte partiture di Luigi Nono (cit. postpraeludium e Prometeo). Ovvero, il compositore non � pi� slegato da una realt� percettiva e teatrale del produzione sonora, ma diventa artefice della disposizione degli altoparlanti, del pubblico e della quantit� di persone che possono simultaneamente partecipare all'esecuzione del pezzo. \\
Vitres de son ha un ascolto ottimale in una configurazione sonor

Sistema di diffusione:
\begin{itemize}
	\item{Scheda Audio 8in 8out}
	\item{Cablaggio}
	\item{2 Amplificatori di potenza da 4 canali a 4 Ohm}
	\item{Computer}
\end{itemize}


