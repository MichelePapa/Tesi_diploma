% !TEX TS-program = pdflatex
% !TEX encoding = UTF-8 Unicode
% !TEX root = ../2018-03-26-papa-vitres-de-son.tex

%************************************************
\chapter{Sinossi}
\label{chp:Sinossi}
%************************************************

	\begin{flushright}
		\textit{Nella nostra anima c'\`e una incrinatura che, se sfiorata, \\
		risuona come un vaso prezioso riemerso dalle profondit\`a della terra} \\
		Wassilly Kandinsky - \emph{Lo Spirituale nell'Arte}
	\end{flushright}

\textit{Vitres de Son - Come un rosone nel cuore di un tempio immenso} non è solo una composizione legata alla creazione e alla ricerca su un nuovo \textit{oggetto sonoro}, ma fa parte di un percorso compositivo e creativo che dura ormai da quattro anni.

Il titolo e il sottotitolo dell'opera sono rispettivamente il titolo e un verso di due poesie del drammaturgo, poeta e attore Antonin Artaud: \textit{Vitres de Son} e \textit{In Sogno}. Entrambe le poesie sono presenti in \textit{Poesie della crudeltà}\footnote{Antonin Artaud, \textit{Poesie della crudeltà} (a cura di P. Di Palmo), Stampa alternativa, Roma 2002, a cura di P. Di Palmo. Pubblicata per la prima volta nel 1925}.

Artaud era artista ai margini, esponente del movimento surrealista e molto discusso dalla critica per le sue idee estremiste riguardanti il teatro, la messa in scena  e la modalità di diffusione della sua arte:

\begin{small}
\begin{quotation}
{\sf Nell'epoca di confusione in cui viviamo, epoca colma di bestemmie e delle fosforescenze di un rinnegamento infinito, in cui tutti i valori sia artistici che morali sembrano sprofondare in un abisso senza altro esempio in nessun epoca dello spirito, ho avuto la debolezza di credere che avrei potuto fare un teatro, che avrei potuto almeno avviare il tentativo di ridare vita al valore universalmente disprezzato del teatro, ma la stupidità di alcuni, la malafede e la spregevole canaglieria di altri me ne hanno distolto per sempre. [...]}\footnote{Antonin Artaud, \textit{Il teatro e il suo doppio}, Einaudi Autore, Roma 1968}
\end{quotation}
\end{small}

L'artista spinge verso una critica sociale che lo porterà ai margini della società e nella quale mi rispecchio. Inoltre, la figura del drammaturgo si può accostare a quella dei \textit{clerici vagantes}\footnote{Furono così chiamati per tutto il sec. XII e il XIII quei poeti, che, vivendo al margine della chiesa, vagavano per le università, le città e le corti, spesso confusi con i giullari, di cui condividevano la vita errabonda e l'indole artistica. [...]

\textit{Enciclopedia Treccani} Edizione 2018.}. Sia Artaud che i clierici erano, infatti, personaggi solitari, irrequieti, artisti vaganti che risentirono quel potente risveglio intellettuale e politico della loro epoca (in entrambe le epoche), rispecchiandone le condizioni sociali, ma soprattutto la fisionomia morale. \\
Artaud ha ricevuto non poche cure psichiatriche dopo essere arrivato a deliri tali da fermarlo anche nella sua produzione artistica, per fortuna, le poesie utilizzate per la composizione, fanno parte del primo periodo artistico, quello giovanile, dove il poeta ha ancora la sensibilità di esprimere un proprio universo immaginifico. Il suo è un paesaggio fatto di personaggi e suoni, potrei azzardare quasi un \textit{paesaggio sonoro}, alla maniera di Schafer, dove ogni oggetto, ogni suono, può diventare protagonista.

\begin{quotation}
{\sf [...] come un rosone nel cuore di un tempio immenso. \\
E là ascolteremo la cadenza immortale \\
delle linee e dei corpi ritmati \\
e di gotiche balaustre profumate \\
dalla dolcezza dei corpi amati \\
dagli uomini con grandi anime cadenzate, \\
\centerline{dai poeti profumati\footnote{Antonin Artaud, \textit{In Sogno} in Poesie della crudeltà}.}}
\end{quotation}

Ecco, qui entra in gioco la fase di analisi e la costruzione compositiva: l'unione di una visione - all'esterno - ascoltando, ad esempio, delle risonanze create da dita sui del rosone nel cuore del tempio immenso: riecheggiano - all'interno \footnote{qui faccio riferimento a \textit{Punto e linea nel piano} di Kandisky} di un ambiente vuoto.
Al margine delle balaustre, assieme ad Artaud e ai clerici vagantes, osserviamo i cambiamenti che avvengono sul linguaggio e sulla sua forma: sull'ambiente.

\begin{quotation}
{\sf \textbf{Vitres de Son} \\
Vitres de son où virent les astres \\
verres où cuisent les cerveaux \\
le ciel fourmillant d'impudeurs \\
dévore la nudité des astres. \\ \\
Un lait bizarre et véhément \\
fourmille au fond du firmament \\
un escargot monte et dérange \\
la placidité des nuages. \\ \\
Délices et rages, le ciel entier \\
lance sur nous comme un nuage \\
un tourbillon d'ailes sauvages \\
torrentielles d'obscénités\footnote{Artaud, \textit{Vitres de Son} in Poesie della crudeltà}.}
\end{quotation}

I \textit{Vetri di suono} diffondono un materiale acustico fascinoso sia nella forma di diffusione che nella tipologia del gesto. \\
Sp.i.r.e.\footnote{L'identità dello strumento verrà resa nota nel capitolo successivo} è l'oggetto sonoro, metafora dei vetri di suono. L'eccitazione delle armoniche e la componente elettronica presenti nel timbro dello strumento, ci spingono verso l'alto, la vibrazione delle sub-armoniche di grandi molle \footnote{vedi cap. 2 \\} ci terrà con i piedi ben saldi al pavimento antistante alla cattedrale, astrazione immaginifica della sala da concerto. \\
Davanti a questo universo fatto di paesaggi e personaggi sonori, entra in causa la composizione. Il processo compositivo legato alle immagini e al ritmo dei versi di Artaud sarà il fulcro dominante dell'andamento formale di \textit{Vitres de Son - Come un rosone nel cuore di un tempio immenso}.
