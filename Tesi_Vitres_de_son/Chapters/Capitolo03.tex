% !TEX TS-program = pdflatex
% !TEX encoding = UTF-8 Unicode
% !TEX root = ../ArsClassica.tex

%************************************************
\chapter{La Performance}
\label{chp:La Performance}
%************************************************

Le difficoltà che intercorrono nel rapporto musicale con il performer sono legate anche al simbolo e alla consequenzialità temporale della notazione. Mi riferisco in particolare al numero di gesti da eseguire sulle molle o sulle placche e alla mancanza di una notazione standard che può far perdere la consequenzialità di tali gesti. \\
Simboli e ritmica, quindi, devono essere puntuali soprattutto nelle parti in cui due gesti o due figure si incastrano tra di loro. Sicuramente, la notazione metronomica e l'utilizzo di accenti è essenziale per la buona riuscita del pezzo, perché, in qualunque caso, si vanno a delineare le micro-forme e la struttura principale del pezzo.

\section{Legame tra esecutore/performer e compositore}
\addcontentsline{toc}{section}{Legame tra esecutore/performer e compositore}


\epigraph{I TEMI \\ Non si tratta di opprimere il pubblico con preoccupazioni cosmiche trascendenti. Che possano esservi chiavi profonde del pensiero e dell'azione in base alle quali leggere tutto lo spettacolo[...]. Tuttavia è necessario che queste chiavi esistano; e la cosa riguarda noi.}{\textit{Antonin Artaud \\ Il teatro e il suo doppio}}

Nel lavoro intercorso con l'esecutore, Matteo Fracassi, studente del dipartimento di Percussioni del Conservatorio Santa Cecilia, ho riscontrato delle difficoltà nel far recepire il contenuto timbrico e dinamico dei gesti. Il legame tra gesto e figura è strettamente correlato, dato che, l'approccio ad uno strumento nuovo ha bisogno di uno studio puntuale sull'utilizzo delle dinamiche e sul capire la correlazione tra il gesto scritto e risultato sonoro. Ogni gesto avrà bisogno di un riscontro sonoro adeguato per evitare che la timbrica stessa di questo nuovo strumento porti a rapportarcisi in modo più improvvisativo che di studio, dalla la natura accattivante del suono in uscita da Sp.I.R.E.. In pratica, ogni gesto rappresentato in partitura sarà la risultante sonora di un determinato timbro o di una determinata ricerca di armoniche possibile sullo strumento. Il performer diventa quindi un essere mitologico, a forma di esecutore, ma con una capacità performativa estesa alla conformazione dello strumento. Una figura complessa che deve avere la capacità di seguire una struttura compositiva che cercherà di essere la più salda possibile e oltremodo precisa. Questo è la parte da esecutore. Quindi, riuscire districarsi nella lettura di una scrittura "quasi" libera. Per quasi, si intende la libertà minima a livello temporale di prendersi le giuste pause, ma allo stesso tempo riuscire a legare i vari fraseggi che si incastrano, restringono, dilatano nel tempo (durata delle frasi) e nello spazio (estensione dello strumento).
