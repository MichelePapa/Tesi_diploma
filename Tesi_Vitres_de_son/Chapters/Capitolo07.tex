% !TEX TS-program = pdflatex
% !TEX root = ../ArsClassica.tex

%************************************************
\chapter{Riflessioni e conclusioni}
\label{chp:Riflessioni e conclusioni}
%************************************************

La strada percorsa fino a questo momento non delinea la fine di un percorso, ma bens� uno step. Un varco importante verso una determinata considerazione degli eventi musicali e dell'evoluzioni del suono.
La strada percorsa fino ad ora non arriva a delineare la fine di un percorso, ma piuttosto un inizio. \\
Questo lavoro di tesi lo identifico di pi� come un percorso costruito su un ponte che ancora sto attraversando che mi spinge sempre pi� verso una consapevolezza artistica. Tale consapevolezza rende giorno per giorno meno vacillanti le fondamenta di questo ponte di transito verso una maturit� stilistica: � la drammaturgia di una struttura musicale in divenire. \\
L'idea che ogni frase, sia ritmica che melodica, risulto come un entit� a s� � il passo odierno e sul quale voglio continuare le mie ricerche formali. Ogni modifica, ogni variazione, � legata ad un determinato personaggio musicale, che si modifica nell'aspetto e nella forma durante ogni cambiamento di oggetti/esseri limitrofi. \\
Anche a livello compositivo cercher� di esprimere tale costrutto: l'identit� di ogni struttura dovr� essere ben salda ed ogni suo mutamento risultare pieno di un proprio significato espressivo, colmo di quella trasformazione intrinseca alle modifiche sul materiale sonoro. Vorrei che il mio studio compositivo futuro sia un lungo viaggio: verso la conoscenza di altre identit� in trasformazione che incontrer� lungo le strade e i percorsi che purtroppo o per fortuna la vita ci propina, sempre con la speranza di una nascita nella quale si veda il sole.

