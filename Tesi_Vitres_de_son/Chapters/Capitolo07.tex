% !TEX TS-program = pdflatex
% !TEX encoding = UTF-8 Unicode

%************************************************
\chapter{Riflessioni e conclusioni}
\label{chp:Riflessioni e conclusioni}
%************************************************

All'inizio di questo percorso accademico non avrei mai immaginato di arrivare fino a questo punto. So di essere cresciuto molto in questi anni, ma considero questa tesi come uno step di un percorso molto più lungo e spero duraturo. Diventare un compositore, soprattutto nell'ambito elettroacustico, è indice di grande maturità e sensibilità.

Ora più che mai, dopo aver creato un strumento elettroacustico, mi sento vicinissimo ad un percorso di ricerca. Aver unito un reverbero a molla e un reverbero a piastra in un unico strumento, mi ha dato molte suggestioni, mi ha indicato altre strade possibili. Il passo successivo sarà sicuramente l'interazione tra Sp.I.R.E. e altri strumenti musicali di liuteria classica. Le ricerche a livello timbrico, meccanico ed elettroacustico, si uniscono ai continui studi musicale, la strada è dubbia e impervia, con degli obbiettivi precisi, ma nel mare magnum delle possibilità nel quale navigare con serenità.

Per il resto, posso solo essere sicuro dei traguardi raggiunti, ovvero, ho trovato una metodologia sia nello studio di materiali acustici, sia di quelli sonori e una lucidità nella creazione di partiture snelle e semplici che danno importanza più alla struttura che a mille stratagemmi notazionali, a volte superflui. 

Chiarezza d'espressione e riproducibilità. Credo che siano questi i due fattori principali per la riuscita di un pezzo. Poi c'è quell'enorme mare di possibilità, di legature, di interventi semantici e prosodici, quell'enorme specchio d'acqua che poeticamente chiamerei musicalità. Forse la vera ricerca sta proprio in questo, nel sentire, tramite le proprie capacità espressive e le tecniche imparate, un'identità musicale.