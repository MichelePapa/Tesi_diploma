% !TEX TS-program = pdflatex
% !TEX encoding = UTF-8 Unicode

%************************************************
\chapter{Riflessioni e conclusioni}
\label{chp:Riflessioni e conclusioni}
%************************************************

Se dovessi delineare ora il percorso artistico da seguire nella mia vita, sicuramente direi: musicale. \\
Anche se, dopo aver creato un oggetto sonoro o strumento che sia (dato che siamo ancora in fase sperimentale non mi azzardo ancora a prendere una strada precisa) mi sento vicinissimo ad un percorso di ricerca.
Aver unito un reverbero a molla e un reverbero a piastra in un unico oggetto sonoro, mi ha dato molte suggestioni, soprattutto, mi ha indicato altre strade possibili, come l'interazione con altri strumenti musicali. A queste ricerche a livello timbrico, meccanico ed elettroacustico, si uniscono i continui studi sulle partiture contemporanee. \\
Se dovessi delineare ora un percorso artistico lineare che volessi seguire, sicuramente direi: musicale. Per il resto, posso solo essere sicuro dei traguardi raggiunti, ovvero, ho trovato una metodologia sia nello studio di materiali acustici, sia di quelli sonori e una lucidità nella creazione di partiture snelle e semplici che danno importanza più alla struttura che a mille stratagemmi notazionali, a volte superflui. \\
Chiarezza d'espressione e riproducibilità. Credo che siano questi i due fattori principali per la riuscita di un pezzo. Poi c'è quell'enorme mare di possibilità, di legature, di interventi semantici e prosodici, quell'enorme specchio d'acqua che poeticamente chiamerei musicalità. Forse la vera ricerca sta proprio in questo, nel sentire, tramite le proprie capacità espressive e le tecniche imparate, qualcosa di musicale. \\
Vorrei, infine, che il mio studio compositivo futuro sia un lungo viaggio: verso la conoscenza di altre identità in trasformazione che incontrerò lungo le strade e i percorsi che la vita ci dona, sempre con la speranza di una nascita nella quale si veda il sole.
