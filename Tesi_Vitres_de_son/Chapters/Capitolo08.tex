% !TEX TS-program = pdflatex
% !TEX encoding = UTF-8 Unicode
% !TEX root = ../2018-03-26-papa-vitres-de-son.tex

%************************************************
\chapter{Bibliografia}
\label{chp:Bibliografia}
%************************************************
Antonin Artaud, \textit{Poesie della crudeltà} (a cura di P. Di Palmo), Stampa alternativa, Roma, 2002. (Pubblicata per la prima volta nel 1925.) \\
\\
Antonin Artaud, \textit{Il teatro e il suo doppio}, Einaudi Autore, Roma, 1968 \\
\\
Walter Branchi, \textit{Tecnologia della musica elettronica} (con prefazione di Domenico Guaccero), Lerici, Roma, 1977 \\
\\
John Cage, \textit{Confessioni di un compositore} in AA.VV. (a cura di G. Bonomo e G. Furghieri), \textit{Riga n. 15} - John Cage, Milano, Marcos y Marcos, Milano, 1998\\ 
\\
Sergio Cingolani e Renato Spagnolo, \textit{Acustica musicale e architettonica}, UTET, Torino, 2004 \\
\\
Guido Facchin,  \textit{Le percussioni}, EDT, 2000 \\
\\
R. Murray Schafer, \textit{The tuning of the world}, Alfred A. Knopf, New York ,1977 \\
\\
R. Murray Schafer, \textit{Il paesaggio sonoro}, Ricordi S.r.l. e LIM Editrice S.r.l., 1985 \\
\\
Samuel Z. Solomon, \textit{How to Write for Percussion: A Comprehensive Guide to Percussion Composition}, Oxford University Press, Gran Britannia, 2016\\
 \\
James Holland, \textit{Practical Percussion: A Guide to the Instruments and Their Sources}, 2005 \\
\\
Curtis Roads, \textit{Composing electronic music, A New Aesthetic} OUP USA, 2015 \\
\\
Curtis Roads, \textit{The Computer music tutorial}, 1996 \\
\\
Tanja Orning \textit{Pression, (a performance study) }, Norwegian Academy of Music, Royal Northern College of Music Vol. 5, 2012 \\
\\
Iannis Xenakis, \textit{Universi del suono, Scritti e interventi 1955-1994} (a cura di Agostino Di Scipio), Ricordi S.r.l. e LIM Editrice S.r.l., 2003 \\
\\
Zaffiri Enore, \textit{Due scuole di musica elettronica in Italia} Silva Editore, Milano, 1968 