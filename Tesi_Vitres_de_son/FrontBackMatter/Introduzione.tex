%!TEX TS-program = pdflatex
%!TEX encoding = UTF-8 Unicode
%!TEX root = ../2018-03-26-papa-vitres-de-son.tex

%*******************************************************
% Introduzione
%*******************************************************

\pdfbookmark{Introduzione}{Introduzione}

Ogni creazione artistica è legata imprescindibilmente sia alla carriera accademica che alla storia personale di chi la scrive. \\
Nel corso della stesura di questa tesi, ho cercato di esprimere il mio processo compositivo e creativo. La forma del testo, per una resa più diretta e scorrevole, è in forma romanzata, è una storia. Cercando in qualche modo di sfuggire da una semplice compilazione di liste sulle tecniche e sui materiali sonori utilizzati. \\
Questa composizione è più un sentiero da intraprendere ed è forse questo il lavoro che un compositore elettroacustico fa giornalmente: non la sola scrittura notazionale e simbolica, ma una continua ricerca di fenomeni che incarnino il proprio vissuto e nel quale un artista può identificarsi, facendoli suoi e riportandoli in musica, creando una propria identità artistica e formale. Come scrisse Kandisky nell'introduzione di \textit{Punto e linea nel piano}: 
\begin{small}
\begin{quotation}
\textit{Di ogni fenomeno si può fare esperienza in due modi. \\ Questi due modi non sono arbitrari ma connessi ai fenomeni -  essi vengono derivati dalla natura dei fenomeni, da due proprietà degli stessi: \\ 
\centerline{esterno - interno\footnote{Wassily Kandisky \textit{Punto e linea nel piano}, Adelphi Edizioni, Roma, 1968}.}}
\end{quotation}
\end{small}


Muovere \textit{interna}mente le corde della propria coscienza e delle proprie esperienze, provando a far vibrare ogni suggestione verso qualcosa di nuovo. Per "qualcosa di nuovo" non intendo un oggetto o una scrittura mai letta o vista prima. Il nuovo, penso, sia semplicemente l'unione di tecniche e di procedimenti fisici che per la prima volta subiscono quella successione di eventi nel tempo. Il tempo crea il nuovo nelle svariate possibilità che la vita ci propone: verso un \textit{esterno} dove abbiamo la possibilità di essere semplici ascoltatori o fautori dei cambiamenti che avvengono in tale realtà. 
